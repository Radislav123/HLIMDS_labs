%!TEX TS-program = xelatex

% Шаблон документа LaTeX создан в 2018 году
% Алексеем Подчезерцевым
% В качестве исходных использованы шаблоны
% 	Данилом Фёдоровых (danil@fedorovykh.ru) 
%		https://www.writelatex.com/coursera/latex/5.2.2
%	LaTeX-шаблон для русской кандидатской диссертации и её автореферата.
%		https://github.com/AndreyAkinshin/Russian-Phd-LaTeX-Dissertation-Template

\documentclass[a4paper,14pt]{article}

\input{data/preambular.tex}
\begin{document} % конец преамбулы, начало документа
    \begin{titlepage}
    \begin{center}
        ФЕДЕРАЛЬНОЕ ГОСУДАРСТВЕННОЕ АВТОНОМНОЕ \\
        ОБРАЗОВАТЕЛЬНОЕ УЧРЕЖДЕНИЕ ВЫСШЕГО ОБРАЗОВАНИЯ\\
        «НАЦИОНАЛЬНЫЙ ИССЛЕДОВАТЕЛЬСКИЙ УНИВЕРСИТЕТ\\
        «ВЫСШАЯ ШКОЛА ЭКОНОМИКИ»
    \end{center}

    \begin{center}
        \textbf{Московский институт электроники и математики}

        \textbf{им. А.Н.Тихонова НИУ ВШЭ}

        \vspace{2ex}

        \textbf{Департамент компьютерной инженерии}
    \end{center}
    \vspace{1ex}

    \begin{center}
        Курс «Высокоуровневое и имитационное моделирование цифровых систем»
    \end{center}


    \begin{center}
        \textbf{ОТЧЕТ\\
        ПО ЛАБОРАТОРНОЙ РАБОТЕ №4
        }
    \end{center}

    \begin{center}
        Тема работы: <<Высокоуровневое моделирование аппаратных проектов.  
        Верификация HDL проектов. DPI, PLI/VPI>>
    \end{center}

    \vspace{2ex}

    \begin{flushright}
        \textbf{Выполнили:}

        \vspace{2ex}

        Студенты группы БИВ174

        Бригада №5

        \vspace{2ex}

        Подчезерцев Алексей Евгеньевич

        Солодянкин Андрей Александрович
        \vspace{2ex}

        \textbf{Принял:}

        асс. МИЭМ НИУ ВШЭ

        Американов А.А.

    \end{flushright}

    \vfill
    \begin{center}
        Москва \the\year \, г.
    \end{center}

\end{titlepage}
\addtocounter{page}{1}
    \tableofcontents
    \pagebreak


    \section{Задание}

    Бригада №5.

    MAX 10 NEEK

    \begin{enumerate}
        \item Ознакомиться с примерами использования DPI и PLI/VPI;
        \item Создать дешифратор 17 разрядный с использованием DPI;
        \item Создать функцию, изменяющую четные числа в массиве на 1.
    \end{enumerate}


    \section{Выполнение работы}

    \subsection{Задание №5}

    Была выполнена программа examples/tutorials/systemverilog/dpi\_basic/.
    Результат представлен на рис.~\ref{fig:05_wave}.
    Сначала включается цвет по умолчанию -- красный, затем включается зеленый, после управление передается в С код.
    В C коде печатается сообщение, включается желтый свет, вызывается функция $sv\_WaitForRed$, которая ожидает 10нс,
    после включается красный свет, управление возвращается в SystemVerilog.
    Через 10нс снова включается зеленый.

    Фрагмент SystemVerilog теста.
    {\small \VerbatimInput{code/05_test.sv}}

    Фрагмент C кода.
    {\small \VerbatimInput{code/05_foreign.c}}

    \begin{figure}[H]
        \centering
        \includegraphics[width=\linewidth]{images/05_wave}
        \caption{Вейвформа для dpi\_basic}
        \label{fig:05_wave}
    \end{figure}

    \subsection{Задание №6}

    Был изучен пример examples/systemverilog/dpi/dpivpipert.
    В данном примере сравниваются подходы PLI/VPI и DPI.

    Первый подход требует большое число кода для генерации и регистрации функций, объявлении стандартных полей.
    DPI же лаконичен и довольно прост, не требует большого числа кода
    DPI является развитием PLI/VPI, логичнее использовать его.

    \subsection{Задание №7}

    Был изучен пример examples/systemverilog/dpi/simple\_calls.

    В данном примере показываются способы вызова C методов из SystemVerilog и наоборот.

    Исходный код С:
    {\small \VerbatimInput{code/07_cimports.c}}

    Исходный код SystemVerilog:
    {\small \VerbatimInput{code/07_simple_calls.sv}}

    Результат выполнения:
    {\small \VerbatimInput{code/07_results.txt}}

    \subsection{Задание №8}

    Был изучен пример examples/systemverilog/dpi/openarray.

    В данном примере демонстрируются различные способы обработки массивов различной размерности.
    Данный подход был использован в работе.

    Результат выполнения:
    {\small \VerbatimInput{code/08_logs.txt}}

    \subsection{Задание №9}

    Был изучен пример examples/systemverilog/dpi/packed\_types.

    Результат выполнения:
    {\small \VerbatimInput{code/09_packed_logs.txt}}

    Был изучен пример examples/systemverilog/dpi/unpacked\_types.

    Результат выполнения:
    {\small \VerbatimInput{code/09_unpacked_logs.txt}}

    При передаче неупакованных данных требуется дополнительно совершать операции упаковки и распаковки.

    \subsection{Задание №10}

    Был изучен пример examples/systemverilog/dpi/checkpoint.

    Данная технология может применяться для выполнения теста с некоторого ненулевого этапа,
    например -- для пропуска длительных этапов.

    \subsection{Задание №11}

    Был изучен пример examples/systemverilog/dpi/cpackages.

    В данном примере изучаются различные способы выполнения команд и проверки их статуса.

    \subsection{Задание №12}

    Был изучен пример examples/systemverilog/dpi/create\_sv\_dynarray.

    Данная технология может применяться для работы с массивами неизвестной начальной длины.


    \section{Самостоятельная работа}

    \subsection{Мультиплексор}

    Была выполнена программа examples/systemverilog/dpi/mux81.
    Корректный результат работы представлен на рис.~\ref{fig:02_mux_ok}.

    \begin{figure}[H]
        \centering
        \includegraphics[width=\linewidth]{images/02_mux_ok}
        \caption{Корректная работа мультиплексора}
        \label{fig:02_mux_ok}
    \end{figure}

    Далее были поданы некоторые сигналы равные b или z.
    Результат работы представлен на рис.~\ref{fig:02_mux_bz}.
    Вывод программы:
    {\small \VerbatimInput{code/02_mux_bz.txt}}

    \begin{figure}[H]
        \centering
        \includegraphics[width=\linewidth]{images/02_mux_bz}
        \caption{Подача b и z на мультиплексор}
        \label{fig:02_mux_bz}
    \end{figure}

    После была внесена ошибка в обработку сигналов мультиплексора.
    Результат работы представлен на рис.~\ref{fig:02_mux_error}.
    Вывод программы:
    {\small \VerbatimInput{code/02_mux_error.txt}}

    \begin{figure}[H]
        \centering
        \includegraphics[width=\linewidth]{images/02_mux_error}
        \caption{Ошибка мультиплексора}
        \label{fig:02_mux_error}
    \end{figure}

    \subsection{Дешифратор}

    В соответствии с заданием был разработан дешифратор 17 разрядный.
    Вейвформа представлена на рис.~\ref{fig:02_decoder}.

    Исходный код C:
    {\small \VerbatimInput{../decoder/decoder.c}}

    Исходный код Verilog модуля:
    {\small \VerbatimInput{../decoder/decoder17.v}}

    Исходный код тестирования Verilog модуля:
    {\small \VerbatimInput{../decoder/decoder17_test.v}}

    \begin{figure}[H]
        \centering
        \includegraphics[width=\linewidth]{images/02_decoder}
        \caption{Результат работы декодера}
        \label{fig:02_decoder}
    \end{figure}

    Результат работы программы:
    {\small \VerbatimInput{code/02_decoder_ok.txt}}

    В исходный код была внесена ошибка, появились отличия в работе программы.
    {\small \VerbatimInput{code/02_decoder_error.txt}}

    \subsection{Последовательность Фибоначчи}

    Была выполнена программа examples/systemverilog/dpi/fibonacci.
    Корректный результат работы представлен на рис.~\ref{fig:03_fib}.

    \begin{figure}[H]
        \centering
        \includegraphics[width=\linewidth]{images/03_fib}
        \caption{Корректная работа последовательности Фибоначчи}
        \label{fig:03_fib}
    \end{figure}

    В программу была внесена ошибка, после чего появились множественные ошибки:
    {\small \VerbatimInput{code/03_fib_error.txt}}

    Далее функция была переписана в рекурсивном варианте:
    {\small \VerbatimInput{../fibonacci/fibonacci.c}}

    Полученный вариант прошел тесты.

    \subsection{Работа с массивами}

    Необходимо изменить исходный массив таким образом, что все четные элементы заменяются 1.
    Вейвформа представлена на рис.~\ref{fig:04_extra}.
    Можно заметить баг вейвформы -- в консоли результат отображается корректно.

    Исходный код C:
    {\small \VerbatimInput{../extra/extra.c}}

    Исходный код тестирования модуля:
    {\small \VerbatimInput{../extra/extra_test.v}}

    \begin{figure}[H]
        \centering
        \includegraphics[width=\linewidth]{images/04_extra}
        \caption{Результат работы программы}
        \label{fig:04_extra}
    \end{figure}

    Результат работы программы:
    {\small \VerbatimInput{code/04_extra.txt}}


    \section{Исходные коды}

    Исходные коды доступны на \href{https://github.com/AsciiShell/hse_hlimds_labs}
    {https://github.com/AsciiShell/hse\_hlimds\_labs}.

    Pull request работы \href{https://github.com/AsciiShell/hse_hlimds_labs/pull/4}
    {https://github.com/AsciiShell/hse\_hlimds\_labs/pull/4}.


    \section{Выводы по работе}

    В ходе работы был изучены технологии DPI и PLI/VPI, их отличия.
    Был получен опыт написания C методов, передачи данных и вызовов C кода из SystemVerilog и наоборот.
    Были написаны самостоятельно модули для верификации работы некоторых устройств и проверки их работы на наличие ошибок.

\end{document} % конец документа
